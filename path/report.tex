\documentclass[a4paper, 11pt]{article}

\usepackage{amsmath}

\title{Using path planning \& inverse kinematics to play chess with UMI-RTX robot arm}
\author{Jouke van der Maas (10186883)  \& Wessel Klijnsma (10172432)}
\date{\today}

\begin{document}
\maketitle

\section{Introduction}
During the second and third week of the Zoeken, Sturen en Bewegen project, a UMI-RTX robot arm
had to be programmed to move chess pieces across a chessboard.  A path planning module and a 
inverse kinematics module were needed to fulfill this task. 

\section{Solution}

	\subsection{Path planning}
	
	\subsection{Inverse kinematics}\label{sec:invkin}
	Inverse kinematic of position deals with the problem: \textit{at which angle should the robot 
	joints be set in order to reach a given position?}. Although there is no general solution to this 
	problem, there are methods that solve a simplified version of it. The UMI-RTX has a 2-link arm, 
	in this case one of the methods gives workable solutions. \\
         The equation that has to be solved is thoroughly described in the syllabus \cite{ldorst01} 
         Summary of the solution:
        
         \begin{equation}
         \begin{bmatrix}
        c_{12} & -s_{12} & 0 & l_1c_1+l_2c_{12} \\
        s_{12} & c_{12} & 0 & l_1s_1 + l_2s_{12} \\
        0 & 0 & 1 & 0 \\
        0 & 0 & 0 & 1
        \end{bmatrix} =
        \begin{bmatrix}
        r_{11} & r_{12} & r_{13} & x \\
        r_{21} & r_{22} & r_{23} & y \\
        r_{31} & r_{32} & r_{33} & z \\
        0 & 0 & 0 & 1
        \end{bmatrix}
        \end{equation}
        
       The solution is :
       
       \begin{equation}
       \theta_1 =  atan2(y, x) - atan2(l_1s_2, l_1 + l_2c_2)
       \end{equation}
       
       \begin{equation}
       \theta_2 = atan2(s_2, c_2)
       \end{equation}
       
       There are two problems with this solution: \textit{redundancy} and \textit{degeneracy}. \\
       There are two solutions to $s_2$, a negative and a positive one, also known as a 'left-hand' 
       and a 'right-hand' solution. This means that for each situation one of these two has to be 
       chosen, this is called \textit{redundancy}. \\
       When position (0,0,0) is to be reached the \textit{degeneracy} occurs. $c_1 = -1$, $s_1 = 0$ 
       and $\theta_2 = - \pi$ which gives $\theta_1 = atan2(0,0) - atan2(0,0)$ which is undefined.  
        
           
\section{Implementation}

	\subsection{Path planning}
	
	\subsection{Inverse kinematics}
	Once the solution described in section \ref{sec:invkin} was found, the implementation of it was
	rather straightforward. To calculate the shoulder and elbow joint values the formulas just had to 
	applied. To deal with the redundancy problem there had to be chosen between left-hand of 
	right -hand configuration. This decision was made based on whether the angle calculated is 
	reach able for the UMI-RTX (shoulder range: -90 - 90) if not, a which between left and right 
	would be made.
	
	
\section{Testing}

	\subsection{Path planning}
	
	\subsection{Inverse kinematics}
	Testing of the inverse kinematics (IK) module was done by comparing its results to results of 
	the standard IK module. This way it was easy to see if the joint values were correct 
	without using the robot. The difference between the created IK module and the standard one 
	is that the standard IK module seems to have different criteria to switch between left and right - 
	hand configuration but this hasn't caused any problems.

\section{Conclusion}

\bibliographystyle{plain}
\bibliography{bib}    
\end{document}
